%% start of file `template.tex'.
%% Copyright 2006-2013 Xavier Danaux (xdanaux@gmail.com).
%
% This work may be distributed and/or modified under the
% conditions of the LaTeX Project Public License version 1.3c,
% available at http://www.latex-project.org/lppl/.


\documentclass[11pt,a4paper,sans]{moderncv}        % possible options include font size ('10pt', '11pt' and '12pt'), paper size ('a4paper', 'letterpaper', 'a5paper', 'legalpaper', 'executivepaper' and 'landscape') and font family ('sans' and 'roman')

% modern themes
\moderncvstyle{banking}                            % style options are 'casual' (default), 'classic', 'oldstyle' and 'banking'
\moderncvcolor{blue}                                % color options 'blue' (default), 'orange', 'green', 'red', 'purple', 'grey' and 'black'
%\renewcommand{\familydefault}{\sfdefault}         % to set the default font; use '\sfdefault' for the default sans serif font, '\rmdefault' for the default roman one, or any tex font name
%\nopagenumbers{}                                  % uncomment to suppress automatic page numbering for CVs longer than one page

% character encoding
\usepackage[utf8]{inputenc}                       % if you are not using xelatex ou lualatex, replace by the encoding you are using
%\usepackage{CJKutf8}                              % if you need to use CJK to typeset your resume in Chinese, Japanese or Korean

% adjust the page margins
\usepackage[scale=0.75]{geometry}
%\setlength{\hintscolumnwidth}{3cm}                % if you want to change the width of the column with the dates
%\setlength{\makecvtitlenamewidth}{10cm}           % for the 'classic' style, if you want to force the width allocated to your name and avoid line breaks. be careful though, the length is normally calculated to avoid any overlap with your personal info; use this at your own typographical risks...

\usepackage{import}

% personal data
\name{Nabil}{Ch}
\title{Curriculum Vitae}                               % optional, remove / comment the line if not wanted
\address{658 Fairbanks St, B1 Condo, Fairbanks, Alaska, AK 99709}{}{}% optional, remove / comment the line if not wanted; the "postcode city" and and "country" arguments can be omitted or provided empty
\phone[mobile]{+1 907 750 6631}                   % optional, remove / comment the line if not wanted
%\phone[fixed]{01234 123456}                    % optional, remove / comment the line if not wanted
%\phone[fax]{+3~(456)~789~012}                      % optional, remove / comment the line if not wanted
\email{ch.nabil.nahin@gmail.com}                               % optional, remove / comment the line if not wanted
%\homepage{www.myname.webs.com}                         % optional, remove / comment the line if not wanted
%\extrainfo{additional information}                 % optional, remove / comment the line if not wanted
%\photo[64pt][0.4pt]{picture}                       % optional, remove / comment the line if not wanted; '64pt' is the height the picture must be resized to, 0.4pt is the thickness of the frame around it (put it to 0pt for no frame) and 'picture' is the name of the picture file
%\quote{Some quote}                                 % optional, remove / comment the line if not wanted

% to show numerical labels in the bibliography (default is to show no labels); only useful if you make citations in your resume
%\makeatletter
%\renewcommand*{\bibliographyitemlabel}{\@biblabel{\arabic{enumiv}}}
%\makeatother
%\renewcommand*{\bibliographyitemlabel}{[\arabic{enumiv}]}% CONSIDER REPLACING THE ABOVE BY THIS

% bibliography with multiple entries
%\usepackage{multibib}
%\newcites{book,misc}{{Books},{Others}}
%----------------------------------------------------------------------------------
%            content
%----------------------------------------------------------------------------------
\begin{document}
%\begin{CJK*}{UTF8}{gbsn}                          % to typeset your resume in Chinese using CJK
%-----       resume       ---------------------------------------------------------
\makecvtitle

\small{Passionate about science, with strong technical, and interpersonal skills. Comfortable working individually or as part of a team.}

\section{Previous Employment}

\vspace{4pt}

\begin{itemize}

\item{\cventry{December 2015--July 2016}{Executive Product Research and Development}{Quartel Infotech Ltd.}{Dhaka}{}{}}

%\vspace{6pt}

\item{\cventry{February 2015--October 2015}{Telecom Engineer, Service provider}{Robi Axiata Ltd.}{Dhaka}{}{}}

%\vspace{6pt}

\item{\cventry{December 2013--January 2015}{Product Development Officer}{Quartel Infotech Ltd.}{Dhaka}{}{}}

\end{itemize}

\section{Education}

\vspace{2pt}

\subsection{Academic Qualifications}

%\vspace{3pt}

\begin{itemize}

\item{\cventry{2016--2018}{Master of Science }{University of Alaska Fairbanks}{Fairbanks}{\textit{Electrical Engineering , CGPA 3.86/4.0}}{}}

\item{\cventry{2009--2013}{Bachelor of Science }{American International University-Bangladesh}{Dhaka}{\textit{Electrical and Electronic Engineering , CGPA 3.86/4.0}}{}}

\end{itemize}

\vspace{2pt}

\subsection{Notable Projects}

%\vspace{3pt}

\begin{itemize}

\item{\textbf{Masters Project:} \textit{'Vehicle detection and speed and length estimation using magnetic sensors'}}

\vspace{2pt}

%\small{As part of my masters thesis, I designed and implemented a system to detect and to estimate the speed and length of the vehicle.}}

%\newpage
\vspace{2pt}

\item{\textbf{NASAs Biologic Analog Science Associated with Lava Terrain (BASALT) group project:}\textit{'Low-Power Extra-planetary Navigation System'}}

%\vspace{3pt}

%\small{We designed a system to help astronauts guide in the right direction and warn them of obstacles in their path, in a low visibility and mobility condition for NASAs BASALT project. My contribution to this project was to detect and warn astronauts of any obstacles in their path using ultrasonic sensors.}}

\vspace{2pt}

\item{\textbf{Wireless sensor networks group project:} \textit{'Distributed Microphone Array for Mobile Robot Localization'}}

\vspace{2pt}

%\small{For this project, we estimated localization of a mobile robot node using time difference of arrival (TDOA) of audio signals and with RSSI (Received Signal Strength Indication) measurements. My responsibilities included RSSI based localization and implementing TDMA MAC protocol for the wireless sensor network.}}

%\vspace{3pt}

\item{\textbf{Embedded Systems group project:}\textit{'General Purpose Bio-Monitoring System'}}

%\vspace{3pt}

%\small{In this project, we collected different physiological signals like ECG, EMG and SPO2 from the subject. My responsibilities for this project included collecting ECG signals and processing the signal to accurately estimate heart rate.}}

\end{itemize}

\section{Technical and Personal skills}

\vspace{2pt}

\begin{itemize}

\item \textbf{Programming Languages:} Proficient in: C, C++, Matlab \\ Also basic ability with: Assembly, VHDL, TeX.

\vspace{2pt}

\item \textbf{Industry Software Skills:} Matlab, Code Composer Studio, Git, Most MS Office products.

\vspace{2pt}

%\item \textbf{General Business Skills:} Good presentation skills, Works well in a team.

%\vspace{6pt}

\item \textbf{Other:} Good soldering skills, Can write well organized and structured reports.

\end{itemize}

\section{Interests and extra-curricular activity}

\vspace{2pt}

\begin{itemize}

\item{Eighth in "Freshmen Programming Contest Fall 2009-2010", organized by AIUB.}

\vspace{2pt}

\item{Participant in “Intra AIUB Programming Contest 2010”, organized by AIUB.}

\vspace{2pt}

\item{Third in AIUB cyber gaming (FIFA) contest 2012.}

\vspace{2pt}

\item{Finalist in AIUB Football Tournament 2011.}

\end{itemize}

%\newpage
%\section{References}

%\vspace{3pt}
 
%\begin{itemize}

%\item{References available on request}

%\end{itemize}

% Publications from a BibTeX file without multibib
%  for numerical labels: \renewcommand{\bibliographyitemlabel}{\@biblabel{\arabic{enumiv}}}% CONSIDER MERGING WITH PREAMBLE PART
%  to redefine the heading string ("Publications"): \renewcommand{\refname}{Articles}
\nocite{*}
\bibliographystyle{plain}
\bibliography{publications}                        % 'publications' is the name of a BibTeX file

% Publications from a BibTeX file using the multibib package
%\section{Publications}
%\nocitebook{book1,book2}
%\bibliographystylebook{plain}
%\bibliographybook{publications}                   % 'publications' is the name of a BibTeX file
%\nocitemisc{misc1,misc2,misc3}
%\bibliographystylemisc{plain}
%\bibliographymisc{publications}                   % 'publications' is the name of a BibTeX file

%-----       letter       ---------------------------------------------------------

\end{document}


%% end of file `template.tex'.
